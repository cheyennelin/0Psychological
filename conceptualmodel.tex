%% Copyright 2009 Ivan Griffin
%
% This work may be distributed and/or modified under the
% conditions of the LaTeX Project Public License, either version 1.3
% of this license or (at your option) any later version.
% The latest version of this license is in
%   http://www.latex-project.org/lppl.txt
% and version 1.3 or later is part of all distributions of LaTeX
% version 2005/12/01 or later.
%
% This work has the LPPL maintenance status `maintained'.
% 
% The Current Maintainer of this work is Ivan Griffin
%
% This work consists of the files tcp_state_machine.tex

%Description
%-----------
%tcp_state_machine.tex - an example file illustrating the TCP (RFC 793)
%                        state machine

%Created 2009-11-20 by Ivan Griffin.  Last updated: 2009-11-20
%-------------------------------------------------------------



\documentclass{article}

\usepackage{tikz}
%%%<
%\usepackage{verbatim}
%%%>

\usepackage{times}  %Required
\usepackage{helvet}  %Required
\usepackage{courier}  %Required

\usetikzlibrary{calc,backgrounds}
\usepackage[active,tightpage]{preview}


\begin{preview}
\begin{tikzpicture}

  %
  % Styles for states, and state edges
  %
  \tikzstyle{state} = [draw, very thick, fill=white, rectangle, minimum height=3em, minimum width=7em, node distance=8em, font={\LARGE}]
  \tikzstyle{stateEdgePortion} = [black,thick];
  \tikzstyle{stateEdge} = [stateEdgePortion,->];
  \tikzstyle{edgeLabel} = [pos=0.5, text centered, font={\LARGE}];

  %
  % Position States
  %
  \node[state, name=user] {User};
  \node[state, name=aspect, below of=user] {Prominent Aspect};
  \node[state, name=evaluation1, below of=aspect, right of=aspect, xshift=6em] {\shortstack{Evaluation on \\ Non-prominent Aspects}};
  \node[state, name=evaluationp, below of=aspect, left of=aspect, xshift=-6em] {\shortstack{Evaluation on \\ Prominent Aspect}};
  \node[state, name=overall, below of=aspect, node distance=14em] {Overall Evaluation};

  %
  % Connect States via edges
  %
  \draw ($(user.south) + (-0.5em,0)$) 
      edge[stateEdge] node[edgeLabel, xshift=-3em]{\emph{Preference $\mathbf{u}$}} 
      ($(aspect.north) + (-.5em,0)$); 
 

  \draw ($(aspect.south) + (-1em,0)$) 
      edge[stateEdge, bend left=22.5] node[edgeLabel, xshift=-2em, yshift=1em]{} 
      ($(evaluationp.east) + (0,1em)$);
  \draw ($(aspect.south) + (1em,0)$) 
      edge[stateEdge, bend right=22.5] node[edgeLabel, xshift=1em, yshift=1em]{}
      ($(evaluation1.west) + (0,1em)$);

  \draw  ($(user.west) + (0,-.5em)$)
      edge[stateEdge, bend left=-45] node[edgeLabel,xshift=-4em]{\emph{Strength $\theta$}} 
     ($(evaluationp.north) + (.5em,0)$) ;


  \draw ($(user.east) + (0,-.5em)$) 
      edge[stateEdge, bend left=45] node[edgeLabel,xshift=1em]{\emph{User-defined Threshold $\mathbf{b}_u$}} 
      ($(evaluation1.north) + (-.5em,0)$);

  \draw ($(evaluationp.east) + (0,-1em)$) 
      edge[stateEdge, bend left=22.5] node[edgeLabel, xshift=-1em, yshift=-1em]{} 
      ($(overall.north) + (-1em,0)$);
  \draw ($(evaluation1.west) + (0,-1em)$) 
      edge[stateEdge, bend right=22.5] node[edgeLabel, xshift=3em, yshift=-1em]{} 
      ($(overall.north) + (1em,0)$);

\end{tikzpicture}
\end{preview}



\end{document}
