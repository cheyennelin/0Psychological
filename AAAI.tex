\def\year{2019}\relax
%File: formatting-instruction.tex
\documentclass[letterpaper]{article} %DO NOT CHANGE THIS
\usepackage{aaai19}  %Required
\usepackage{times}  %Required
\usepackage{helvet}  %Required
\usepackage{courier}  %Required
\usepackage{url}  %Required
\usepackage{graphicx}  %Required
\frenchspacing  %Required
\setlength{\pdfpagewidth}{8.5in}  %Required
\setlength{\pdfpageheight}{11in}  %Required
%PDF Info Is Required:
  \pdfinfo{
/Title ()
/Author (AAAI Press Staff)}
\setcounter{secnumdepth}{0}  
 \begin{document}
% The file aaai.sty is the style file for AAAI Press 
% proceedings, working notes, and technical reports.
%
\title{Non-Compensatory Psychological Models for Recommender Systems}
\author{ID: 335
}
\maketitle
\begin{abstract}
The study of consumer psychology reveals two categories of procedures used by consumers to make consumption related choices: compensatory rules and non-compensatory rules. Existing models assume the consumers follow the compensatory rules, which are to make decisions based on a weighted or summated score over different aspects. Our main contribution in this paper is to improve performance of recommender systems by adopting non-compensatory rules to make ranking decisions.  We present non-compensatory versions for three most commonly adopted ranking models in this area, i.e. BPR, BTL and SVD++. We show that, the non-compensatory versions all outperform their original models on a wide range of real data sets. 
\end{abstract}



\section{Introduction}


\section{Related Work}

\section{Ranking Models for Recommendations}
\subsection{Ranking Aware Loss Functions}
%pairwise loss functions

%pointwise loss functions


\subsection{Modeling }
\section{Non-Compensatory Ranking Models for Recommendations}

\section{Experiments}

\end{document}
