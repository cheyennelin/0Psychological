% Author : Alain Matthes
% Source : http://altermundus.com/pages/examples.html
\documentclass{article}

\usepackage[utf8]{inputenc}
\usepackage[upright]{fourier}
\usepackage{tikz}
\usetikzlibrary{matrix,arrows,decorations.pathmorphing}
\usepackage[active,tightpage]{preview}
 \PreviewEnvironment{tikzpicture}
% l' unite
\newcommand{\myunit}{1 cm}
\tikzset{
    node style sp/.style={draw,circle,minimum size=\myunit},
    node style ge/.style={circle,minimum size=\myunit},
    arrow style mul/.style={draw,sloped,midway,fill=white},
    arrow style plus/.style={midway,sloped,fill=white},
}
\begin{document}
%\begin{preview}
\begin{tikzpicture}
% les matrices
\matrix (U) [matrix of math nodes,%
             nodes = {node style ge},%
             left delimiter  = (,%
             right delimiter = )] at (0,0)
{%
\node(u1) {\mathbf{u}_{1}};%
&\ldots
         & \node(uk){\mathbf{u}_{k}};%
                  & \ldots%
                           & \node(uK) {\mathbf{u}_{K}}; \\
};
\node [draw,below=10pt] (labelu) at (U.south) 
    { $U$ : user preference vector};

\matrix (V) [matrix of math nodes,%
             nodes = {node style ge},%
             left delimiter  = (,%
             right delimiter =)] at (6*\myunit,6*\myunit)
{%
  \node(v1){\mathbf{q}_{1}} ;\\
\vdots \\
   \node(vk) {\mathbf{q}_{k}};  \\
 \vdots\\
                 \node(vK) {\mathbf{q}_{K}} ; \\
};
\node [draw,above=10pt] at (V.north) 
    { $V$ : item feature vector};

\node (C) at (6*\myunit,0) {$\mathbf{R}_{u,q}$} ;

\draw[<->,red](u1) to[in=180,out=90]
	node[arrow style mul] (x) {$\mathbf{u}_{1}\times \mathbf{q}_{1}$} (v1);
\draw[<->,red](uk) to[in=180,out=90]
	node[arrow style mul] (y) {$\mathbf{u}_{k}\times \mathbf{q}_{k}$} (vk);
\draw[<->,red](uK) to[in=180,out=90]
	node[arrow style mul] (z) {$\mathbf{u}_{K}\times \mathbf{q}_{K}$} (vK);

\draw[red,->] (x) to node[arrow style plus] {$+\raisebox{.5ex}{\ldots}+$} (y)%
                  to node[arrow style plus] {$+\raisebox{.5ex}{\ldots}+$} (z)%
                  to (C.north west);
\node [draw,below=20pt] at (C.south) 
    { $R$: weighted summation};
\end{tikzpicture}
%\end{preview}
\end{document}
